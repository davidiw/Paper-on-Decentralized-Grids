\documentclass{sig-alternate}

\usepackage{natbib}
\setlength{\bibsep}{0.0pt}

\usepackage{url}
\usepackage{multirow}
\usepackage{array}
\usepackage{epsfig}
\usepackage{footnote}
\usepackage{amsmath}

\setlength{\parskip}{0pt}
\setlength{\parsep}{0pt}
\setlength{\headsep}{0pt}
\setlength{\topskip}{0pt}
\setlength{\topmargin}{0pt}
\setlength{\topsep}{0pt}
\setlength{\partopsep}{0pt}
\setlength{\floatsep}{5pt}
\setlength{\textfloatsep}{5pt}
\setlength{\intextsep}{0pt}
%\setlength{\abovecaptionskip}{0pt}
%\setlength{\belowcaptionskip}{0pt}

%\widowpenalty=100000
%\clubpenalty=100000

\begin{document}

\title{On the Use of Group-Oriented Autonomic VPNs for High-Throughput Computing}

\author{
David Isaac Wolinsky and Renato Figueiredo
\\
University of Florida
\\
}

\maketitle

\begin{abstract}
High-throughput computing (HTC) attempts to acquire and use as many resources,
both dedicated and opportunistic, as possible.  In this paper, we explore the
challenge of connecting resources across administrative domains from different
institutions and into users' homes.  Several works discuss the usage of
virtualization and middleware as means to connect and secure resources.  In
this paper, we present methods that automate the configuration of resources
through group-based Web 2.0 interfaces and connect the resources through a
peer-to-peer (P2P) based virtual private network (VPN) creating secure ad-hoc,
decentralized, and distributed computing grids supporting both experts and
non-experts alike.  We verify our approach quantitatively by using our
reference implementation to connect resources distributed across institutions
and compute clouds to determine the time for the system to become fully active
and for a single user to submit and complete jobs on all the resources.
\end{abstract}

\section{Introduction}
High throughput computing (HTC) is a form of opportunistic computing that,
unlike high performance computing (HPC) seeking only powerful resources,
benefits from ad-hoc, arbitrary resources including computers found in computer
labs, homes, and offices as well as cloud resources and retired HPC clusters.
Creating and maintaining HTC systems that cross administrative domains (grids)
require expertise in networks, operating systems, and grid middleware to
configure the sites uniformly and guarantee connectivity amongst sites involved.

Administrators each have their own way of configuring resources and may be
hesitant or unwilling to configure resources in a way that conflicts with the
environmental norm.  This conflicts with the requirements of merging resources
across administrative domains as computing resources are typically configured
uniformly having a common environment and network rules.  Administrators would
prefer not having exceptional rules for a subset of resources under their
domain and do not want to grant access to lesser known, remote users.  Network
constraints such as firewalls and network address translation (NAT) can prevent
cross-domain communication.  With exceptions this may be ameliorated, but
additional rules may be required for each new cluster or resource added to the
cross-domain grid.

HTC clusters are not limited to large systems requiring dedicated
administrators: there are many systems that discuss the use of desktop or
opportunistic grids, such as Boinc~\cite{boinc} and PVC~\cite{pvc}.  While Boinc
solutions may be easily configured, the approach relies on a centralized
scheduler and that applications be compiled using Boinc API.  While PVC
provides enables parallel tasks and more decentralized system configuration,
the approach has scalability concerns and relies on a centralized node to
assist in node connectivity.  In general, there exists no solution that
provides scalable, self-configuring, decentralized grid systems for non-experts.

Connecting resources distributed across the Internet can be challenging due to
limit of IP (Internet Protocol) addresses available to an organization.  NAT
further complicates the issue by limiting the formation of direct links between
remote sites without external assistance.  When creating HTC systems across a
small set of institutions, approaches that use Internet Service Provider VPNs,
Layer 2 Tunneling Protocol (L2TP) VPNs, and other user-configured VPN
approaches can be used.  Most solutions require some form of centralization
and static links; as systems expand dynamically, the manual configuration of
the system grows significantly.  VPNs do, however, provide means of securing
the system and, through the use of proper middleware, can allow users to
interact with each others resources without additional configuration from the
local site administrator.

The Archer project~\cite{archer}, a collaborative academic environment
linking institutions and external users for the purpose of computer architecture
research, is an example of a real system having the described constraints.  Many
researchers have a need for resources occasionally and rather than investing
in a large pool of dedicated resources for a single institution they are able
to pool their resources together using HTC mechanisms.  The resulting system
allows individuals the opportunity to complete their jobs quickly and ensure
that their resource contribution is not idle when locally unused.  This use
case potentially introduces a new issue where the users may not be experts nor
have the ability to include an expert in the construction of the system.

Explicitly the requirements for a system in these environments are: 1) users
should be able to easily add and remove resources, 2) resources should not
require configuration to allow remote users access, 3) tasks that run inside
the grid should not have access to external resources, 4) external resources
should not be able to access the grid, 5) resource priority should be granted
to the resource's owner, and 6) malicious users should be able to be removed.
In this paper, we describe our approach to handling these requirements to
enable the creation of a dynamic, decentralized HTC grid through a novel
approach involving decentralized overlays enabling a self-configuring VPN
and HTC environment.  

In previous work, we bundled IPOP~\cite{ipop} for decentralized virtual
networking and other grid middleware into a virtual machine called the
Grid Appliance~\cite{grid_appliance}.  The approach was highly coupled, and
while it made it easy to connect to existing grids to add or access resources,
it required expertise for users to create and manage their own independent
grids.  This paper extends that work to enable non-experts to create and manage
their own grids through a group-oriented model embodied in a web 2.0
infrastructure providing a public key infrastructure along with VPN and grid
configuration.  The approach describes methods that can be used to easily
to configure and combine resources from virtual, physical, and cloud
environments.

The rest of this paper is organized as follows.  Section~\ref{middleware}
surveys core grid middleware, while Section~\ref{implementation} describes
the methods by which we enable user-friendly creation and management of grids.
A qualitative comparison of our system to other approaches is presented in
Section~\ref{related_work}.  We perform quantitative evaluations on our system
and present our findings in Section~\ref{evaluation}.  Finally, in
Section~\ref{conclusion}, we conclude with a discussion on real systems using
our approach.

\section{Core Grid Middleware}
\label{middleware}
This section surveys existing grid middleware solutions presenting existing
software solutions that address the constraints of our system.

\subsection{Task Schedulers}
The most fundamental requirement of a cluster is the task scheduler.  Each
task scheduler has a general focus and selecting one that works well in a
specific environment can make the configuration of the system significantly
easier.  Generally, there are three approaches to configure HTC clusters: 1)
task workers pull from a centralized manager as employed by Boinc~\cite{boinc},
2) task workers receive jobs from a centralized submission site, and 3) task
workers receive jobs from any member of the HTC infrastructure.  Both 2 and 3
can be implemented using a myriad of different job schedulers with verifying
levels of difficulty.  Task schedulers supporting this behavior include
Condor~\cite{condor0}, Sun Grid Engine, and PBS and its relative Torque.

Unlike other task schedulers, Condor supports decentralized users supported
by having separate
components that keep track of resources and negotiates resource allocation from
those that make the resource requests and submit tasks.  This abstraction
allows for a simple centralized system to maintain the grid without requiring
any run-time configuration.  In addition, Condor allows open unauthenticated
access to the grid as long as a peer is within a subnet, using a VPN ensures
that that only members of the VPN have access to grid resources.  To enable this
behavior in other grid schedulers would require modification or additional
middleware, like Globus~\cite{globus_toolkit}.  Other reasons motivating the
use of Condor include it being open source, having an active community, and
focus on opportunistic cycles.  In the list of requirements, Condor also
handles the ability to assign groups or institutions privilege on their own
resources when shared in a collaborative environment.  Condor supports
multiple, decentralized negotiators through flocking.

%In full length paper compare other task schedulers...
%Though settling on a task scheduler, we feel it is important to present a
%comparison of other features provided by task schedulers, as illustrated in
%Table~\ref{table:task_schedulers}.  Experience dictates that the approach
%that requires the least amount of work to obtain the most amount of features is
%ideal.  What we learned during our reading of manuals and experimental
%configurations was that Condor can easily be configured to provide a P2P-like
%scheduling environment, whereas other approaches were heavily centralized.
%While they could be decentralized and each had their own benefits, it is our
%opinion, that it did not outweigh the ease of use provided by Condor.

\subsection{P2P VPNs}
Many of the requirements described can be addressed through the use of a
VPN.  A VPN can assist in dealing with connecting network constrained resources,
securing the grid from the outside world, and removing malicious forces inside.
Grid computing has seen its fair share of VPNs such as ViNe~\cite{vine},
Violin~\cite{violin}, and VNET~\cite{vnet}.  These approaches are limited by
their lack of self-configuration, namely that static links between peers must
be created, security credentials must be manually distributed, and lack of
support for direct connectivity between NAT and firewalled peers without
additional configuration from the user limiting their applicability for such
resources to communicate with each other through proxy servers.  In
PVC~\cite{pvc}, the authors describe an approach that self-configures through
a centralized server with NAT traversal support, which works on many
NAT devices but only when used without a stateful firewall.  The drawbacks with
this approach are the centralized key distribution center (KDC) and lack of
encrypted links.

IPOP~\cite{ipop} provides a P2P virtual network with decentralized and
self-configuring link creation with NAT traversal support that works with
most NATs using a distributed hash table (DHT) for address
allocation and resolution.  Previous approaches~\cite{grid_appliance} used
IPsec for security or went without it entirely as IPOP lacked the ability to
secure links.  In~\cite{groupvpn}, we presented GroupVPN, which transforms IPOP
into an automated VPN with enhanced NAT handling through the use of
decentralized relays (proxies), enabling two-hop, low latency connections when
NAT and firewall traversal fails.

The authentication system employs a public key infrastructure (PKI), made
accessible through a group-based Web 2.0 environment.  Users can create and
join groups, and group owners can grant user access, promote users to
administrative capabilities, or remove users who have overstayed their
welcome.  A PKI has a very natural P2P aspect, in that, peers can mutually
authenticate each other by verifying signatures on the exchanged certificates
unlike the centralized authentication such as a KDC.  To automatically configure
the system, users download a GroupVPN configuration file through the group
website, which can be provided to the GroupVPN by command-line or GUI.  The
next time GroupVPN is started, it will use this configuration to automatically
obtain a signed certificate by sending a certificate request along with a
shared secret contained in the configuration to the group server through HTTPS,
uniquely authenticates the user.  If the user has appropriate permissions, the
server will sign the certificate request.  To remove peers, the system supports
a reliable, centralized user revocation list located at the group website and
decentralized revocation by broadcast and distributed data stores.

\section{Implementation and Design}
\label{implementation}
Middleware alone does not make a grid, the system must be configured for the
components to work with each other.  This section discusses a solution to this
issue using a group infrastructure to create and manage a grid with packaged
environments allowing users to configure their own resources independent of
pre-existing virtual machine images.

\subsection{GroupAppliances}
An appliance is defined as a dedicated, blackbox device requiring little if any
configuration from the user.  While traditional computing appliances like a
router, network storage, or even cluster resources have been available as
hardware appliances, the recent resurgence of virtualization initiated by
VMware and Xen has made software appliances by means of virtual appliances
popular.  Recently, cloud computing has become popular in large part thanks to
Amazon's EC2.  Both virtual and cloud resources present themselves as cheap
computing for HTC and opportunistic computing purposes, because they can be
setup in such a way as to have no or limited effect on users' computers and can
be shutdown when no longer in use.  Even with virtual and cloud appliances 
available to tap into these resources, they still require some manual
manual configuration to form a grid, and these packaged solutions cannot easily
be applied to hardware resources.

Our solution is the creation of a generic software stack that self-configures
based upon a user input configuration file.  The contents of a configuration
file are the type of resource (dedicated compute node, job scheduling node, a
mixture of the two, or the job negotiating central server); the user's group and
username on the site; and the grid's GroupVPN configuration data.  The
configuration file is generated from a Web 2.0 group-based infrastructure
called GroupAppliance, using a single GroupVPN group to connect all members of
the grid together in a VPN but using the GroupAppliance group to distinguish
their resource contributions.  Thus many GroupAppliances groups can be linked
together through a GroupVPN group.  By distinguishing resource contributions,
users are able to get credit and gain priority to their resources when
submitting tasks to run.

When a user downloads the configuration file from the GroupAppliance
infrastructure, the data is stored in a floppy disk image that can be used on
physical resources by writing the image to a real floppy disk or to a USB drive,
a VM by adding a virtual disk image to the VM, and clouds through instance
specific configuration data.  EC2 provides per-cloud instance configuration in a
parameter called ``user data'' providing up to 16 KB of data available only to
that cloud instance.  At the time of this writing, it appears that EC2 is the
only cloud provider to offer this option.  Alternatively, users could configure
an image specifically to run for this cloud by inserting the floppy image into
the cloud image and then generating cloud instances from this new image, or the
user could setup a storage cloud where the cloud instances could retrieve the
floppy.  Because the floppy contains private GroupVPN configuration data it
should not stored on public resources.

Upon booting, the grid configuration scripts parses the floppy to determine how
to configure the machine.  Negotiators insert an ad into the DHT, whereas
resources and task submitters query the DHT for the list of negotiators,
selecting one and relegating the rest for flocking.  At which point, tasks
can be submitted and run.

\subsection{Constructing Environments}
Often VMs are favored for the distribution of complicated applications as
experts can configure them and release the results as a complete working system.
This approach may limit non-expert use to the VM appliance, which may be
undesirable for users that want to configure their own systems without reuse
of the existing VM.  Guides may exist for the creation of systems, most
systems are too complex for non-experts to produce similar results found in the
VM.  Alternatively, we have moved towards the use of packages (DEB and RPM)
that can be installed in arbitrary environments and through the use of package
managers (APT and YUM) handle configuration such that the requirements listed
in the introduction are handled.  Packages can be provided that automatically
install and configure the task scheduling middleware and a VPN as well as
sandbox the environment limiting users to accessing the virtual network and not
external resources such as other local resources and the Internet.  The
remaining components are configurable through the GroupAppliances interface and
decentralized through the DHT.

The most important components in securing an environment are limiting internal
and external access from inside the system.  Specifically, internal resources
should avoid password enabled accounts to avoid cases where users submit tasks
that attempt to provide more privileged access to the user.  In the event that
this is not possible, ensuring that the task scheduler runs as user without
the ability to switch permissions can prevent this attack, for example,
preventing \textit{su} access in Linux.  By default, Condor will run jobs as
either nobody or the user named ``Condor''.  This limits access to many of the
core components of the system already, but it does not limit the users ability
to read files that allow reading from anyone on the system and the ability to
communicate to external resources from inside the machine.  Making the user
data directory readable only by the user and group who own the files and
directories will prevent remote users from reading local user personal data.
Limiting access to external resources can be implemented via a firewall,
allowing the Condor user to only have the ability to send packets over the
virtual network.

For job submission nodes, there is one more consideration, we attempt to make
these as user-friendly as possible.  To do this, we have enabled file system
and remote access capabilities for the machines so that a user on the same
host can access the resources without having to configure additional utilities
or using the VMs interface, if available.  Though, we do not want users to
access the data through the virtual network or the Internet from a public
address, with regard to virtual machines, we create a second network card that
supports host-only connectivity.  By binding all user applications to use the
network interface, they do not require extra security enhancements.
Alternatively, applications like SSH could be enabled to only allow private key
based login.  In our model, we expect only dedicated compute nodes and possibly
the job negotiator running on physical or cloud resources and user nodes to
use VMs.

To prepare a system is straight-forward: users configure the package manager to
link to a package distribution site and then install the desired packages for
the grid resource.  When finished, the user can restart the device or restart
the grid service with the floppy image adding a new resource to the grid.  The
VPN will acquire a signed certificate and grid configuration scripts will
configure Condor and other services through interaction with the DHT.

\section{Related Work}
\label{related_work}
There are many projects that seek to provide easy to use resources for HTC
and opportunistic computing.  We focus on two approaches whose focus is
user-friendly dynamic grids: PVC~\cite{pvc} and GPU~\cite{gpu}.

PVC or Private Virtual Clusters creates instant grids using a PVC specific
virtual network and task scheduler as well as VMs to isolate remote jobs from
users' resources.  Resources discover and TCP NAT traversal are performed
through a centralized system broker, though it is unclear how the resources are
configured with the knowledge of the broker, nor how a broker is configured.
Loss of the broker can prevent usability of the system.  PVC virtual network
lacks privacy, links are authenticated through a KDC but messages are not
encrypted or authenticated.  PVC scalability constraints are unclear, as
experiments were limited to 8 nodes.  In contrast, our approach focuses on
privacy, scalability, and self-configuration through a decentralized system.

GPU or Global Processing Unit uses the Gnutella P2P system to create a
completely decentralized computing grid.  The authors state that the expected
size for grids range from 5 to 15 nodes.  While the authors do not mention NAT
traversal, there are many Gnutella systems that do support various forms of it.
There is no mention of providing safety to the users' resources from malicious
tasks.  While GPU provides easy configuration, it lacks the ability to run
jobs in a sandbox and support large pools of resources.

There are many other desktop grid environments that use Boinc as the underlying
method to push jobs.  As we explained earlier, Boinc uses a few approaches that
we found undesirable for our requirements with the primary issue that Boinc job
scheduling is heavily centralized.  In addition, for Boinc systems that allow
running arbitrary applications, it is unclear how secure they are.

\section{Evaluation}
\label{evaluation}
This evaluation focuses on the time required to create and utilize a grid
consisting of various distributed resources built a reference implementation of
the system described in this paper.  Using VMware resources behind a Cisco
and ``iptables'' NAT at the University of Florida, KVM resources behind an
``iptables'' and KVM NAT at Northeastern University, and Cloud resources
provided by EC2, pools of 50 resources from each site were booted independently
and then in parallel.  The resources connected with an existing GroupVPN and
GroupAppliance group with a negotiator and client node already online.  Upon
all the resources connecting, the client submits a job to all resources.  The
time required to deploy the resources starts with the beginning of copying
files and ends when the last VM has been started.  Time for the machines to
connect includes this time up until the time when all of the deployed resources
appear in ``condor\_status''.  Time for connectivity begins when a client
client submits a job to sleep for 5 minutes to every resource and all tasks
have completed.  All tasks are automated through scripts with human interaction
required only to start the events of grid boot and job submission.  Results are
presented in Figure~\ref{fig:results}.

\begin{small}
\begin{figure}[ht]
\setlength{\itemsep}{0pt}
\setlength{\parskip}{0pt}
\centering
\begin{tabular}[c]{|m{1.5cm}||m{1.2cm}|m{1.2cm}|m{1.2cm}|m{1.2cm}|} \hline
& EC2 & NEU & UF & All \\ \hline\hline
Start & 2:44 & 10:21 & 20:23 & 21:14 \\ \hline
Connect & 5:10 & 21:47 & 24:16 & 38:27\\ \hline
Run & 7:15 & 6:35 & 5:53 & 21.19 \\ \hline
\end{tabular}
\caption{Time in seconds to start and connect resources to an existing
grid and run jobs from a client using the stack described by the paper.}
\label{fig:results}
\end{figure}
\end{small}

As the systems consist of various hardware and software configurations, the
time to start is provided as a basis for the remaining numbers.  Some of the
interesting experiences of the experiment were:  the combination of the
``iptables'' and VMware NAT was more easily traversable than the EC2
NAT or the combination of ``iptables'' and KVM NAT, as made explicit by the
time to connect and the time to run; the VPN had some issues handling 150
connections simultaneously, though over time connections for all sites were
formed; and Condor has retry delays that affected results if the tasks
were not accomplished immediately, well represented by the results for ``All''.

\section{Conclusion and Real Use Cases}
\label{conclusion}
In this paper, we presented a novel approach to creating autonomic grid systems
through a configuration data generated and distributed through a user-friendly
Web 2.0 group-based infrastructure providing configuration for both the grid
and the VPN.  The VPN provides a completely decentralized system having no
single point of failure.  Grid resources are pre-configured with only identity
and discover each other through the VPN's DHT.  Users never need to access a
console to create a HTC cluster, an attractive feature for non-experts.  The
evaluation establishes this process as a valid method for connecting distributed
resources in a scalable fashion, but in larger systems, administrators may need
to be patient for all resources to appear.

We have several deployments using the system described in this paper.  Over
the past 2 years, we have had an active grid deployed for computer architecture
research, Archer~\cite{archer}.  Archer currently spans four universities with
500 resources, we have had hundreds of students and researchers submitting jobs
with over 150,000 hours of total job execution in the past year alone.  Groups
at the Universities of Florida, Clemson, Arkansas, and Northwestern Switzerland
have used it as a tool to teach grid computing.  Purdue is constructing a large
campus grid using GroupVPN to connect resources together.  Recently, a grid at
La Jolla Institute for Allergy and Immunology went live with minimal
communication with us.

\bibliographystyle{abbrv}
\small{
\bibliography{DecentralizedGrids}
\suppressfloats
}

\end{document}
